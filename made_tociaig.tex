\documentclass[conference]{IEEEtran}
% If the IEEEtran.cls has not been installed into the LaTeX system files,
% manually specify the path to it: e.g.,
% \documentclass[conference]{../sty/IEEEtran}

\usepackage[latin1]{inputenc}
\usepackage{amssymb}
\setcounter{tocdepth}{3}
\usepackage{graphicx}
\usepackage{subfigure}

\usepackage{url}

% correct bad hyphenation here
\hyphenation{}

\IEEEoverridecommandlockouts    % to create the author's affliation portion
                % using \thanks

\textwidth 178mm    % <------ These are the adjustments we made 10/18/2005
\textheight 239mm   % You may or may not need to adjust these numbes again
\oddsidemargin -7mm
\evensidemargin -7mm
\topmargin -6mm
\columnsep 5mm

\setlength{\textfloatsep}{8pt plus 2pt minus 2pt}
\setlength{\intextsep}{8pt plus 2pt minus 2pt}


%TODO remove this to back to the original format
\usepackage[usenames,dvipsnames]{xcolor}
\usepackage[colorinlistoftodos,prependcaption,textsize=tiny]{todonotes}
\usepackage{soul}


\newcommand{\notajj}[1]{\todo[size=\small,linecolor=green,backgroundcolor=green!90!white,bordercolor=purple,]{#1 \\ -- \footnotesize \color{purple} JJ}}

\newcommand{\notapedro}[1]{\todo[size=\small,linecolor=blue,backgroundcolor=blue!20!white,bordercolor=blue]{#1 \\ -- \footnotesize \color{blue} Pedro}}

\newcommand{\notaantonio}[1]{\todo[size=\small,linecolor=red,backgroundcolor=orange!20!white,bordercolor=red]{#1 \\ -- \footnotesize \color{orange} Antonio}}

\newcommand{\notamaribel}[1]{\todo[size=\small,linecolor=red,backgroundcolor=orange!20!white,bordercolor=red]{#1 \\ -- \footnotesize \color{red} Maribel}}

\definecolor{cvrivas}{rgb}{0.9,0.4,0.4}
\newcommand{\notavrivas}[1]{\todo[size=\small,linecolor=cvrivas,backgroundcolor=cvrivas!20!white,bordercolor=white]{#1 \\ -- \footnotesize \color{cvrivas} Victor}}

\definecolor{cfergu}{rgb}{0.2,0.5,0.9}
\newcommand{\notafergu}[1]{\todo[size=\small,linecolor=cfergu,backgroundcolor=cfergu!20!white,bordercolor=black]{#1 \\ -- \footnotesize \color{cfergu} Fergu}}

\definecolor{cpaloma}{RGB}{152,251,152}
\definecolor{cbpaloma}{RGB}{0,100,0}
\newcommand{\notapaloma}[1]{\todo[size=\small,linecolor=cbpaloma,backgroundcolor=cpaloma!20!white,bordercolor=cbpaloma]{#1 \\ -- \footnotesize \color{cbpaloma} Paloma}}

\definecolor{cgus}{RGB}{239,255,18}
\newcommand{\notagus}[1]{\todo[size=\small,backgroundcolor=cgus,bordercolor=cgus,linecolor=cgus]{#1 \\ -- \footnotesize Gustavo}}

\definecolor{ctares}{rgb}{0.1,0.1,0.1}
\newcommand{\notantares}[1]{\todo[size=\small,linecolor=gray,backgroundcolor=gray!30!white,bordercolor=black]{#1 \\ -- \footnotesize \color{orange} Antares}}

\newcommand{\notaruben}[1]{\todo[size=\small,linecolor=PineGreen,backgroundcolor=SeaGreen!30!white,bordercolor=PineGreen]{#1 \\ -- \footnotesize \color{PineGreen} Ruben}}

\newcommand{\notaimportante}[1]{\todo[size=\large,linecolor=PineGreen,backgroundcolor=SeaGreen!30!white,bordercolor=PineGreen]{#1 \\ -- \footnotesize \color{PineGreen} Ruben}}

\usepackage[paperwidth=297mm,
            paperheight=297mm,
            left=60mm,
            top=50pt,
            textwidth=178mm,
            marginparsep=10pt,
            marginparwidth=50mm,
            textheight=239mm,
            footskip=50mm]
           {geometry}

\begin{document}


% paper title: Must keep \ \\ \LARGE\bf in it to leave enough margin.

\title{\ \\ \LARGE\bf Massive emergence of monomyths in non-player
  characters}

\author{R.H. Garc\'ia-Ortega, P. Garc\'ia-S\'anchez, ... \thanks{Department of Computer Architecture and Computer Technology, University of Granada, Spain, {\tt \{raiben,pablogarcia\}@ugr.es}}}
% avoiding spaces at the end of the author lines is not a problem with
% conference papers because we don't use \thanks or \IEEEmembership
% use only for invited papers
%\specialpapernotice{(Invited Paper)}

% make the title area
\maketitle

\begin{abstract}
Designing backstories for all the non-player characters in Open World videogames and make them different, coherent and interesting for each game is a goal that has not been reached by the state-of-the-art methodologies and technologies. Our objective in this paper is to describe the  design and evaluation of a method to procedurally
create coherent massive backstories for non-player characters where
\textit{the monomyth}, also called \textit{the hero's journey}, has to emerge.
The \textit{monomyth} is a template that conforms to ancient
and modern myths from cultures all over the world and links different characters and archetypes. It is present in novels\notaruben{Lovecraft, por ejemplo http://digitalcommons.liberty.edu/ masters/277/}, films\notaruben{The Monomyth in American Science Fiction Films: 28 Visions of the Hero's Journey} and videogames\notaruben{skyrim, mass effect (http://ses.library.usyd.edu.au/ handle/2123/8902), zelda, etc}, and different researchers use it to model interesting plots in virtual worlds\notaruben{por ejemplo este http://mud.co.uk/richard/ CGAIDE94.pdf}, thus we use it a metric of interest in backstories.
We formally analyze the archetypes present in \textit{the monomyth}
using predicate logic and provide the basis to model facts 
of the world in a declarative paradigm. Then, general constraints for
future implementations are defined. On this basis, we model the
characters as intelligent agents using deterministic finite
automata, and fitness functions using logical deductions on the
world's facts, in a hybrid three-layered \textit{Evolutionary
  Computation - Agent Based Model} approach that optimally
parametrizes the agents. The analysis of the results show that we are
able to obtain the parameters of the agents that, after the execution
in a massive 
multi-agent virtual simulation, better allow the emergence of the
complex patterns present in \textit{the monomyth}.
Then, the events of the world inhabited by these agents can be used as
the backstories of the characters that populate an open world
videogame. The present paper will provide as example a literary
narration of a backstory created with the best parameters found. 
The method described can also be applied to different game settings or
characters' nature by re-implementing the Agent's layer using the
rules provided in this paper. 

\end{abstract}
%%%%%%%%%%%%%%%%%%%%%%%%%%%%%%%%%%%%%%%%%%%%%%%%%%%%%%%%%%%%%%%%%%%%%%%%%%%%%%%

\section{Introduction}
\label{sec:intro}

Videogames have evolved into a mass medium: According to the report by
the Entertainment Software Association (ESA) \cite{esa_ef_2014}, 59\%
of the US citizens play videogames and the total consumer spend on
games in US reaches \$21,53 billions. The computer and video game
industry in US directly and indirectly employs more than 146,000
people \cite{esa_century_2014}. In this very competitive context, the
Procedural Content Generation (PCG) is gaining more and more
popularity in the last few years due to the cost reduction and
re-playability it offers to the developers. To give an example,
\textit{No man's sky}, an indie adventure videogame where planets,
fauna and flora are created procedurally, won three Game Critics
Awards, including Best Original Game and the Special Commendation for
Innovation, in the past Electronic Entertainment Expo 2014, a
reference annual trade show for the video game
industry\footnote{\url{http://www.gamecriticsawards.com/winners.html}}. 

In videogames, non-player characters usually interact with the main
players in order to challenge him, offer information to complete their
goals or
provide life-likeness to the virtual world. In open worlds, a type of videogame where a player can roam freely through a virtual world and thousand of characters coexist, a procedural method to generate
them is mandatory, and the realism of the world will be affected
drastically by it.

Our previous work has confirmed that a hybrid \textit{Evolutionary
  Computation - Agent Based Model} (EC-ABM) methodology can be used to
achieve the emergence of archetypes. In \cite{garcia14my} we proposed
the use of simple agents and behavior patterns and proved that a
Genetic Algorithm could lead to a parametrization of the agents that
made different archetypes emerge. In \cite{garcia2015world} we
analyzed the effect of the model parameters such as
the map size, available food per virtual day or initial number of
agents.
Several conclusions were attained: the existence of conflicts is essential for the emergence of most of our archetypes, different archetypes need different types of conflicts and, in many cases, conflicts are also emergent.
However, the main problem of this methodology for the emergence of
backstories was that no {\em a priori} rules
could be provided to establish whether or not a design of an agent
could lead to the appearance of the archetypes, since no formal models to define the archetypes or the events in the virtual world were used. For example, if the agent has no actions that fire events related to \textit{encouraging}, no archetype \textit{herald} could ever emerge, as we discuss on section~\ref{sec:met}. For this reason, in the present paper, we formally model the archetypes using predicate logic. 
A logical model allows us to describe the archetypes in a way that is 
close to natural language, provides the formalism and syntax of
the mathematical language and is easily transformable to a computer
program in a declarative paradigm. This logical model also allows us
to define the requirements of the ABM layer and, following them, design a
new more complex finite state automaton that better fits the
archetypes needs.

%una �ltima frase (antes de comentar la estructura del paper) diciendo qu� esperamos obtener al modelar formalmente los arquetipos con l�gica de predicados. Una especie de s�ntesis un una frase para que al lector le quede claro lo que pretendemos hacer y para qu�.
 
\notajj{The rest of the paper is organized as follows}

%%%%%%%%%%%%%%%%%%%%%%%%%%%%%%%%%%%%%%%%%%%%%%%%%%%%%%%%%%%%%%%%%%%%%%%%%%%%%%%

\section{State of the Art}
\label{sec:sota}

%%%%%%%%%%%%%%%%%%%%%%%%%%%%%%%%%%%%%%%%%%%%%%%%%%%%%%%%%%%%%%%%%%%%%%%%%%%%%%%

\section{Methodology}
\label{sec:met}

%%%%%%%%%%%%%%%%%%%%%%%%%%%%%%%%%%%%%%%%%%%%%%%%%%%%%%%%%%%%%%%%%%%%%%%%%%%%%%%

\section{Experimental setup}
\label{sec:exp}

%%%%%%%%%%%%%%%%%%%%%%%%%%%%%%%%%%%%%%%%%%%%%%%%%%%%%%%%%%%%%%%%%%%%%%%%%%%%%%%

\section{Experiments and Results}
\label{sec:res}

%%%%%%%%%%%%%%%%%%%%%%%%%%%%%%%%%%%%%%%%%%%%%%%%%%%%%%%%%%%%%%%%%%%%%%%%%%%%%%%

\section{Conclusions}

%%%%%%%%%%%%%%%%%%%%%%%%%%%%%%%%%%%%%%%%%%%%%%%%%%%%%%%%%%%%%%%%%%%%%%%%%%%%%%%

\section{Acknowledgments}


\bibliographystyle{IEEEtran}
\bibliography{garcia_ortega}

\newpage
\listoftodos[Notes]

\end{document}
%%% Local Variables:
%%% ispell-local-dictionary: "english"
%%% hunspell-local-dictionary: "english"
%%% End: