\documentclass[conference]{IEEEtran}
% If the IEEEtran.cls has not been installed into the LaTeX system files,
% manually specify the path to it: e.g.,
% \documentclass[conference]{../sty/IEEEtran}

\usepackage[latin1]{inputenc}
\usepackage{amssymb}
\setcounter{tocdepth}{3}
\usepackage{graphicx}
\usepackage{subfigure}

\usepackage{url}

% correct bad hyphenation here
\hyphenation{}

\IEEEoverridecommandlockouts    % to create the author's affliation portion
                % using \thanks

\textwidth 178mm    % <------ These are the adjustments we made 10/18/2005
\textheight 239mm   % You may or may not need to adjust these numbes again
\oddsidemargin -7mm
\evensidemargin -7mm
\topmargin -6mm
\columnsep 5mm

\setlength{\textfloatsep}{8pt plus 2pt minus 2pt}
\setlength{\intextsep}{8pt plus 2pt minus 2pt}


%TODO remove this to back to the original format
\usepackage[usenames,dvipsnames]{xcolor}
\usepackage[colorinlistoftodos,prependcaption,textsize=tiny]{todonotes}
\usepackage{soul}


\newcommand{\notajj}[1]{\todo[size=\small,linecolor=green,backgroundcolor=green!90!white,bordercolor=purple,]{#1 \\ -- \footnotesize \color{purple} JJ}}

\newcommand{\notapedro}[1]{\todo[size=\small,linecolor=blue,backgroundcolor=blue!20!white,bordercolor=blue]{#1 \\ -- \footnotesize \color{blue} Pedro}}

\newcommand{\notaantonio}[1]{\todo[size=\small,linecolor=red,backgroundcolor=orange!20!white,bordercolor=red]{#1 \\ -- \footnotesize \color{orange} Antonio}}

\newcommand{\notamaribel}[1]{\todo[size=\small,linecolor=red,backgroundcolor=orange!20!white,bordercolor=red]{#1 \\ -- \footnotesize \color{red} Maribel}}

\definecolor{cvrivas}{rgb}{0.9,0.4,0.4}
\newcommand{\notavrivas}[1]{\todo[size=\small,linecolor=cvrivas,backgroundcolor=cvrivas!20!white,bordercolor=white]{#1 \\ -- \footnotesize \color{cvrivas} Victor}}

\definecolor{cfergu}{rgb}{0.2,0.5,0.9}
\newcommand{\notafergu}[1]{\todo[size=\small,linecolor=cfergu,backgroundcolor=cfergu!20!white,bordercolor=black]{#1 \\ -- \footnotesize \color{cfergu} Fergu}}

\definecolor{cpaloma}{RGB}{152,251,152}
\definecolor{cbpaloma}{RGB}{0,100,0}
\newcommand{\notapaloma}[1]{\todo[size=\small,linecolor=cbpaloma,backgroundcolor=cpaloma!20!white,bordercolor=cbpaloma]{#1 \\ -- \footnotesize \color{cbpaloma} Paloma}}

\definecolor{cgus}{RGB}{239,255,18}
\newcommand{\notagus}[1]{\todo[size=\small,backgroundcolor=cgus,bordercolor=cgus,linecolor=cgus]{#1 \\ -- \footnotesize Gustavo}}

\definecolor{ctares}{rgb}{0.1,0.1,0.1}
\newcommand{\notantares}[1]{\todo[size=\small,linecolor=gray,backgroundcolor=gray!30!white,bordercolor=black]{#1 \\ -- \footnotesize \color{orange} Antares}}

\newcommand{\notaruben}[1]{\todo[size=\small,linecolor=PineGreen,backgroundcolor=SeaGreen!30!white,bordercolor=PineGreen]{#1 \\ -- \footnotesize \color{PineGreen} Ruben}}

\newcommand{\notaimportante}[1]{\todo[size=\large,linecolor=PineGreen,backgroundcolor=SeaGreen!30!white,bordercolor=PineGreen]{#1 \\ -- \footnotesize \color{PineGreen} Ruben}}

\usepackage[paperwidth=297mm,
            paperheight=297mm,
            left=60mm,
            top=50pt,
            textwidth=178mm,
            marginparsep=10pt,
            marginparwidth=50mm,
            textheight=239mm,
            footskip=50mm]
           {geometry}

\begin{document}


% paper title: Must keep \ \\ \LARGE\bf in it to leave enough margin.

\title{\ \\ \LARGE\bf Massive emergence of monomyths in non-player
  characters} \notajj{Quito el genitivo y adem�s open worlds
  entero. No se entiende en el contexto}

\author{R.H. Garc�a-Ortega, P. Garc\'ia-S\'anchez, ... \thanks{Department of Computer Architecture and Computer Technology, University of Granada, Spain, {\tt \{raiben,pablogarcia\}@ugr.es}}}
% avoiding spaces at the end of the author lines is not a problem with
% conference papers because we don't use \thanks or \IEEEmembership
% use only for invited papers
%\specialpapernotice{(Invited Paper)}

% make the title area
\maketitle

\begin{abstract}

In videogames, non-player characters usually interact with the main players in order to combat, offer information to complete goals or provide realism and aliveness to the virtual world. In open worlds where thousand of characters coexist, a procedural method to generate them is mandatory, and the realism of the world will be affected drastically by it.

In the present work, we design and evaluate a method to procedurally create coherent backstories for non-player characters where \textit{the monomith}, a complex behavior pattern that links different characters and archetypes also called \textit{the hero's journey}, has to emerge. We formally analyze the archetypes present in \textit{the monomith} using predicate logic and provide the basis to model facts of the world in a declarative paradigm. Then, general constraints for future implementations are defined. On this basis, we model the characters as intelligent agents using deterministic finite automatons, and fitness functions using logical deductions on the world's facts, in a hybrid three-layered \textit{Evolutionary Computation - Agent Based Model} approach that optimally parametrizes the agents.

The analysis of the results show that we are able to obtain the parameters of the character that, after the execution in a massive multi-agent virtual simulation, better allows the emergence of the complex patterns present in \textit{the monomith}. The agents generated in each execution can be used as characters that populate an open world in a videogame, providing backstories that the user may perceive as interesting and unique. This method can also be applied to different game settings or characters' nature by re-implementing the Agent's layer using the rules provided in this paper.\notaruben{Quiza me enrollo mucho en las implicaciones, aunque creo que resume los objetivos desde el punto de vista de aplicacion. Por supuesto, tambien hablo de los resultados, asi que estar�a bien revisarlo cuando los tengamos para ver si es verdad}

\end{abstract}
%%%%%%%%%%%%%%%%%%%%%%%%%%%%%%%%%%%%%%%%%%%%%%%%%%%%%%%%%%%%%%%%%%%%%%%%%%%%%%%

\notaruben{Titulo provisional... muchos genitivos sajones... Que alguien creativo haga un titulo impactante :-)}

\section{Introduction}
\label{sec:intro}
Videogames have evolved into a mass medium. According to the report by the Entertainment Software Association (ESA) \cite{esa_ef_2014}, 59\% of the US citizens play videogames and the total consumer spend on games in US reaches \$21,53 billions. The computer and video game industry in US directly and indirectly employs more than 146,000 people \cite{esa_century_2014}. In this very competitive context, the Procedural Content Generation (PCG) is gaining more and more popularity in the last few years due to the cost reduction and re-playability it offers to the developers. To give an example, \textit{No man's sky}, an indie adventure videogame where planets, fauna and flora are created procedurally, won three Game Critics Awards, including Best Original Game and the Special Commendation for Innovation, in the past Electronic Entertainment Expo 2014, a reference annual trade show for the video game industry\footnote{\url{http://www.gamecriticsawards.com/winners.html}}.



%%%%%%%%%%%%%%%%%%%%%%%%%%%%%%%%%%%%%%%%%%%%%%%%%%%%%%%%%%%%%%%%%%%%%%%%%%%%%%%

\section{State of the Art}
\label{sec:sota}

%%%%%%%%%%%%%%%%%%%%%%%%%%%%%%%%%%%%%%%%%%%%%%%%%%%%%%%%%%%%%%%%%%%%%%%%%%%%%%%

\section{Methodology}
\label{sec:met}

%%%%%%%%%%%%%%%%%%%%%%%%%%%%%%%%%%%%%%%%%%%%%%%%%%%%%%%%%%%%%%%%%%%%%%%%%%%%%%%

\section{Experimental setup}
\label{sec:exp}

%%%%%%%%%%%%%%%%%%%%%%%%%%%%%%%%%%%%%%%%%%%%%%%%%%%%%%%%%%%%%%%%%%%%%%%%%%%%%%%

\section{Experiments and Results}
\label{sec:res}

%%%%%%%%%%%%%%%%%%%%%%%%%%%%%%%%%%%%%%%%%%%%%%%%%%%%%%%%%%%%%%%%%%%%%%%%%%%%%%%

\section{Conclusions}

%%%%%%%%%%%%%%%%%%%%%%%%%%%%%%%%%%%%%%%%%%%%%%%%%%%%%%%%%%%%%%%%%%%%%%%%%%%%%%%

\section{Acknowledgments}


\bibliographystyle{IEEEtran}
\bibliography{garcia_ortega}

\newpage
\listoftodos[Notes]

\end{document}
